\section{On test environment}
\label{appendix:testenv}
All the test we presented were carried out in a protected environment, in order
to avoid performance fluctuation due to system load. A 8 GB RAM machine running
ArchLinux has been prepared, with a idle memory usage of 300 MB (base system and
OpenSSH server). The processor is an Intel i5-6200U, 2 core and 4 threads due to
HyperThreading, with a processor base frequency of 2.3 GHz. The Linux processor
scheduler is performance, which fix the frequency to 2.7 GHz.\\

To solve to optimality some big instances we take advantage of The NEOS Server
\citep{czyzyk1998neos}, where a concorede-CPLEX solver has been installed.


\section{On parameters tuning}
\label{appendix:tuning}
We decided to not present all the parameters that we tune during the tests. We
may be shared them in percentage but almost never in absolute values. The reason is
that they're too related to the test environment and to the instance size and to
the time limit used, that is 10 minutes for each instances tested. If an
explanation of a parameter is missing (GRASP's $k$, for example), the answer we
should give is \say{it is reasonable given the context} and we can't explain why
because, literally, it's reasonable given the context. If one would argue
against a slightly higher or lower value, well, we took for that reason.
Clearly, we reported in the thesis a range or an average value for that specific
parameter.\\
This works for parameters, strategies are always explained.

\section{On code}
Code can be found at \href{https://github.com/lugot/tsp-approximation}{https://github.com/lugot/tsp-approximation} 
